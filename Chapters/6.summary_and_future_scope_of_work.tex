% Chapter Template

\chapter{Conclusion and Future Work} % Main chapter title

\label{Chapter6} % For referencing the chapter elsewhere, use \ref{Chapter6}

%----------------------------------------------------------------------------------------

This chapter summarizes the key findings of our investigation into fine-tuning failures on the MS MARCO passage ranking task, discusses the broader implications for the information retrieval community, and outlines promising directions for future research.

\section{Summary of Key Findings}

\subsection{Universal Fine-Tuning Degradation}

Our comprehensive investigation provides definitive evidence that conventional fine-tuning approaches consistently fail to improve upon heavily optimized baseline performance on the MS MARCO passage ranking task. All five model variants exhibited performance degradations ranging from 13.5\% to 32.3\% in MRR@10 compared to the base model.

This finding fundamentally challenges the conventional wisdom that fine-tuning universally improves model performance, particularly when working with extensively pre-trained models that have already undergone domain-specific optimization with billion-scale data.

\subsection{Evidence for the Saturation Phenomenon}

Our results provide compelling evidence for the saturation phenomenon: MS MARCO represents a benchmark where our base model has achieved near-optimal performance through extensive domain-specific pre-training. The model's substantial prior MS MARCO exposure during billion-scale training means that additional task-specific training introduces destructive noise rather than beneficial signal.

This finding has profound implications for how the research community approaches fine-tuning on established benchmarks where models may have already seen substantial portions of the data during pre-training.

\subsection{Embedding Space Degradation as Primary Failure Mode}

Through systematic UMAP visualization analysis, we demonstrated that embedding space degradation serves as the primary mechanism underlying fine-tuning failure. This degradation follows a predictable pattern: from structured semantic organization in the base model (achieved through billion-scale contrastive learning \cite{gao2021simcse}) to complete uniformity in the worst-performing variants.

The strong correlation between embedding space structure and retrieval performance validates our diagnostic methodology and suggests that geometric analysis should become standard practice in fine-tuning research.

\subsection{The Hard Negatives Paradox}

Contrary to conventional wisdom in information retrieval, hard negatives consistently harmed performance across all model architectures. This paradox becomes understandable when considering that the base model has already seen extensive MS MARCO data during pre-training. Our hard negatives introduced conflicting signals that disrupted the sophisticated semantic understanding already encoded during the model's billion-scale training phase.

This finding suggests that negative sampling strategies should be fundamentally reconsidered when working with extensively pre-trained models.

\subsection{Hidden Costs of Parameter-Efficient Methods}

Our investigation revealed two critical limitations of LoRA for retrieval tasks:
\begin{enumerate}
\item \textbf{Catastrophic sensitivity} to hard negatives when applied to saturated models
\item \textbf{Unexpected computational overhead} during inference, with approximately 2× slower performance despite parameter efficiency
\end{enumerate}

These findings challenge assumptions about LoRA's deployment advantages and suggest that parameter efficiency does not guarantee computational efficiency, particularly when the base model is already highly optimized.

\section{Theoretical Contributions}

Our work establishes a theoretical framework for understanding scale disparity effects between pre-training and fine-tuning phases, explaining why billion-scale pre-training creates robust baselines that focused fine-tuning cannot improve upon. We also introduced a systematic methodology for diagnosing fine-tuning failures through embedding space analysis, providing tools to understand model behavior beyond traditional metrics.

\section{Practical Implications}

Our findings suggest practitioners should consider pre-training exposure when evaluating fine-tuning effectiveness, recognize that extensively pre-trained models may already provide optimal performance, and incorporate embedding space analysis into evaluation pipelines. Model developers should focus on architectural improvements rather than parameter tuning for saturated benchmarks.

\section{Limitations}

Key limitations include single dataset focus (MS MARCO), scale constraints (1M vs 1B samples), focus on dual-encoder models excluding cross-encoders, and exclusive reliance on triplet loss. These constraints limit generalizability but provide clear directions for future work.

\section{Future Research Directions}

Future investigations should examine less saturated benchmarks, explore regularization techniques to preserve pre-trained structure, investigate alternative loss functions, and develop hybrid architectures combining sparse and dense retrieval methods. Advanced diagnostic tools for real-time embedding space monitoring would also be valuable.

\begin{itemize}
\item \textbf{Hybrid Architectures:} Combine sparse retrieval methods like BM25 with dense retrieval for improved performance
\item \textbf{Cross-Encoder Integration:} Develop efficient cross-encoder approaches for saturated benchmarks
\item \textbf{Dynamic Adaptation:} Create architectures that can adapt to new domains without disrupting existing knowledge
\end{itemize}

\subsection{Diagnostic Tool Development}

Advanced diagnostic tools should include:

\begin{itemize}
\item \textbf{Real-Time Monitoring:} Develop tools for monitoring embedding space changes during training
\item \textbf{Embedding-Based Early Stopping:} Create stopping criteria based on geometric degradation rather than traditional validation metrics, using UMAP-based clustering quality measures
\item \textbf{Interpretability Tools:} Build systems to explain why specific fine-tuning approaches fail
\end{itemize}

\section{Broader Impact}

Our findings suggest a fundamental paradigm shift from parameter tuning to architectural innovation, emphasizing efficient use of existing knowledge rather than simply scaling up. This investigation demonstrates that conventional wisdom about fine-tuning universality does not hold for extensively pre-trained models on saturated benchmarks.

The challenge of improving upon billion-scale pre-trained models represents a new frontier requiring architectural innovation and improved understanding of model behavior. Our work provides both empirical evidence for fine-tuning limitations and diagnostic tools necessary to address these limitations, establishing a foundation for future research in neural information retrieval systems.