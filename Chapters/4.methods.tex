% Chapter 4

\chapter{Methodology} % Main chapter title

\label{Chapter4} % For referencing the chapter elsewhere, use \ref{Chapter4}

%------------------------------------------------------------------

\section{Reproducibility and Open Science}

To support reproducibility and further research, we provide open access to:

\begin{itemize}
\item \textbf{Source Code:} Complete implementation available at \url{https://github.com/omnikingzeno/ms-marco-fine-tuning-experiments}
\item \textbf{Custom Hard Negatives Dataset:} Available at \url{https://huggingface.co/datasets/manupande21/}
\item \textbf{Fine-tuned Models:} All model variants available at \url{https://huggingface.co/manupande21/}
\item \textbf{Experimental Logs:} Training curves and detailed metrics included in repository
\item \textbf{Visualization Code:} UMAP generation and analysis scripts in repository
\end{itemize}

This chapter presents our experimental methodology for investigating fine-tuning failures on the MS MARCO passage ranking task, detailing dataset construction, model configurations, training procedures, and evaluation protocols.

\section{Dataset and Preprocessing}

\subsection{MS MARCO Dataset Overview}

The MS MARCO passage ranking dataset comprises 8,841,823 passages extracted from web documents, with 1,010,916 training queries and sparse relevance judgments \cite{msmarco}. Each query is associated with one or more relevant passages, creating a challenging retrieval scenario where systems must identify relevant content from a large corpus.

For our experiments, we utilized 808,731 unique training queries, the full passage collection, and 55,578 query-passage relevance judgments for evaluation.

\subsection{Dataset Construction Strategy}

We constructed two distinct training datasets to investigate negative sampling strategy impacts on domain-adapted models:

\subsubsection{Random Negatives Dataset}
We randomly sampled 1 million triplets from the original triples.train.small.tsv file, maintaining query distribution and ensuring diverse negative examples following standard neural ranking practices.

The random sampling process ensured:
\begin{itemize}
\item Uniform distribution across different query types
\item Preservation of original positive-negative ratio patterns
\item Diverse negative examples from across the passage collection
\end{itemize}

\subsubsection{Hard Negatives Dataset}

We constructed 503,000 hard negative triplets using a sophisticated methodology designed to challenge the model with semantically similar but non-relevant passages:

\begin{enumerate}
\item For each training query, we retrieved the top-200 passages using the base model
\item We randomly sampled a passage from ranks 51 to 200 (excluding known positives) as the hard negative
\item This approach ensures negatives are semantically similar but non-relevant, avoiding reliance on BM25 \cite{robertson2009probabilistic} for negative mining
\end{enumerate}

This methodology creates challenging negative examples that are likely to be semantically related to the query but lack the specific relevance captured in the ground truth labels.

\section{Model Architectures and Configurations}

\subsection{Base Model Architecture}

The sentence-transformers/all-MiniLM-L6-v2 model employs a dual-encoder Siamese network architecture \cite{dong2022exploring, bromley1993signature} with the following mathematical formulation:

\begin{equation}
E_q = \text{MiniLM}(q), \quad E_p = \text{MiniLM}(p)
\end{equation}

\begin{equation}
\text{sim}(q,p) = \frac{E_q \cdot E_p}{||E_q|| \cdot ||E_p||}
\end{equation}

where $E_q$ and $E_p$ represent 384-dimensional embeddings for query $q$ and passage $p$ respectively.

The model architecture consists of:
\begin{itemize}
\item 6 transformer layers \cite{vaswani2017attention}
\item 384 hidden dimensions  
\item 12 attention heads
\item 22.7 million total parameters
\item Vocabulary size: 30,522 tokens
\end{itemize}

\subsection{Pre-training Exposure Analysis}

Critically, this model underwent extensive domain-specific fine-tuning using self-supervised contrastive learning objectives \cite{gao2021simcse} on over 1 billion sentence pairs \cite{huggingface_minilm}. The training procedure utilized:

\begin{itemize}
\item \textbf{Hardware:} 7 TPU v3-8 cores for efficient computation
\item \textbf{Pre-training steps:} 100,000 with batch size 1,024
\item \textbf{MS MARCO exposure:} 9,144,553 sentence pairs specifically from MS MARCO
\item \textbf{Learning strategy:} Contrastive learning \cite{gao2021simcse} with cross-entropy loss and learning rate warm-up
\end{itemize}

This extensive pre-training established the model as highly optimized for semantic search tasks, particularly those involving MS MARCO-style query-passage relationships.

\subsection{Model Variants}

Our experimental design encompasses five model configurations to systematically evaluate different fine-tuning approaches:

\begin{enumerate}
\item \textbf{Base SBERT:} Unmodified sentence-transformers/all-MiniLM-L6-v2 (pre-trained on 1B pairs including 9.1M MS MARCO samples)
\item \textbf{Full FT (Random):} Full parameter fine-tuning on 1M random negatives
\item \textbf{Full FT (Hard):} Full parameter fine-tuning on 503K hard negatives
\item \textbf{LoRA FT (Random):} LoRA adaptation (r=16, $\alpha$=32) on 1M random negatives  
\item \textbf{LoRA FT (Hard):} LoRA adaptation (r=16, $\alpha$=32) on 503K hard negatives
\end{enumerate}

\subsection{LoRA Configuration}

The LoRA configuration targets query and value projection matrices in the attention layers, representing approximately 3.9\% of the total model parameters while maintaining expressive capacity for task adaptation. The mathematical formulation follows:

\begin{equation}
W_{q/v}' = W_{q/v} + \Delta W = W_{q/v} + B A
\end{equation}

where $B \in \mathbb{R}^{384 \times 16}$ and $A \in \mathbb{R}^{16 \times 384}$ are the trainable low-rank matrices with rank $r=16$ and scaling factor $\alpha=32$.

\subsubsection{Parameter Count Calculation}

The 3.9\% figure represents the proportion of trainable LoRA parameters relative to the total model parameters. The detailed calculation follows:

\textbf{Base Model Parameters:}
\begin{itemize}
\item Total parameters in sentence-transformers/all-MiniLM-L6-v2: 22,713,216
\item This includes embedding layers, 6 transformer blocks, and pooling components
\end{itemize}

\textbf{LoRA Trainable Parameters:}

In LoRA fine-tuning, the number of trainable parameters depends on the rank ($r$) and the number of affected matrices. Since LoRA modifies the query and value projection matrices in the multi-head attention mechanism \cite{vaswani2017attention}, we calculate the total parameters when rank = 16.

Each query and value projection matrix in MiniLM-L6-v2 has a shape of $384 \times 384$ per attention head. LoRA replaces these with low-rank matrices of shape $384 \times 16$ and $16 \times 384$, meaning each head contributes:

\begin{equation}
(384 \times 16) + (16 \times 384) = 12,288 \text{ parameters per head}
\end{equation}

Since there are 12 attention heads per layer and 6 layers, the total number of trainable LoRA parameters is:

\begin{equation}
12,288 \times 12 \times 6 = 884,736 \text{ parameters}
\end{equation}

\begin{equation}
\text{Percentage} = \frac{884,736}{22,713,216} \times 100\% = 3.9\%
\end{equation}

This parameter efficiency demonstrates LoRA's ability to achieve task adaptation with minimal parameter overhead while preserving the original model's knowledge.

\subsection{Training Objective: Triplet Loss Selection}

Our choice of triplet loss as the primary training objective was motivated by two key factors that align with our research design and objectives.

\subsubsection{Data Structure Alignment}

Our experimental design explicitly constructs triplets in the form $(q, p^+, p^-)$ where $q$ represents the query, $p^+$ is the positive (relevant) passage, and $p^-$ is the negative (non-relevant) passage. This data structure maps directly to triplet loss formulation, which is specifically designed for scenarios involving anchor-positive-negative relationships. 

The triplet loss objective ensures that the embedding space learns to place positive passages closer to their corresponding queries than negative passages by a specified margin:

\begin{equation}
\mathcal{L}_{triplet} = \max(0, ||E_q - E_{p^+}||^2 - ||E_q - E_{p^-}||^2 + \text{margin})
\end{equation}

where $E_q$, $E_{p^+}$, and $E_{p^-}$ represent the embeddings for query, positive passage, and negative passage respectively.

\subsubsection{Research Focus on Negative Sampling}

Our core research investigation centers on comparing the effectiveness of hard versus random negative sampling strategies on saturated models. Triplet loss provides an ideal framework for this comparison because it incorporates an explicit negative example in each training instance, allowing direct evaluation of negative quality and its impact on model performance.

This design choice was essential for discovering the "hard negatives paradox" where semantically similar but non-relevant passages (hard negatives) actually harm performance more than randomly sampled negatives when applied to extensively pre-trained models. Alternative loss functions that aggregate multiple negatives or use different sampling strategies would have obscured this critical finding.

\section{Infrastructure and Computational Setup}

\subsection{Hardware Configuration}

All experiments were conducted on a remote system provided by Modal.com \cite{modal2023} featuring:
\begin{itemize}
\item NVIDIA A100 80GB GPU
\item 1.2TB RAM
\item High-bandwidth network connectivity
\end{itemize}

This high-performance setup provided substantial computational capability for our experiments while ensuring reproducible conditions across all model variants.

\subsection{Training Configuration}

Consistent training hyperparameters were maintained across all fine-tuning experiments to ensure fair comparison:

\begin{itemize}
\item \textbf{Optimizer:} AdamW \cite{loshchilov2017decoupled} with $\beta_1=0.9, \beta_2=0.999$
\item \textbf{Learning rate:} 2e-5 with linear warmup (1,000 steps)
\item \textbf{Batch size:} 128 (limited by GPU memory constraints)
\item \textbf{Epochs:} 5 for Full FT (Hard), 3 for all others
\item \textbf{Gradient clipping:} 1.0 to prevent gradient explosion
\item \textbf{Weight decay:} 0.01 for regularization
\end{itemize}

\section{Evaluation Protocol}

\subsection{Performance Metrics}

Performance evaluation employed standard information retrieval metrics:

\begin{itemize}
\item \textbf{MRR@k (Mean Reciprocal Rank at k):} A ranking-based metric that measures the quality of retrieval systems. For each query, MRR@k considers only the top-k retrieved documents and calculates the reciprocal of the rank position of the first relevant document. Formally, for a set of queries Q, MRR@k is defined as:
\begin{equation}
\text{MRR@k} = \frac{1}{|Q|} \sum_{i=1}^{|Q|} \frac{1}{\text{rank}_i}
\end{equation}
where $\text{rank}_i$ is the position of the first relevant document for query $i$ within the top-k results, or 0 if no relevant document appears in the top-k. Higher MRR@k values indicate better retrieval performance. We evaluate at cutoffs k=10 and k=100.

\item \textbf{Inference Time:} Wall-clock time for processing 10,000 queries
\item \textbf{Training Efficiency:} Convergence behavior and training stability
\end{itemize}

\subsection{Inference Time Measurement}

Inference time measurements included:
\begin{itemize}
\item Query encoding time
\item Cosine similarity computation
\item Top-200 passage ranking
\end{itemize}

Excluded from timing measurements:
\begin{itemize}
\item Model loading overhead
\item File I/O operations  
\item Evaluation metric computation
\end{itemize}

\subsection{Embedding Space Analysis}

We employed UMAP visualization to analyze embedding space geometry changes across model variants. The analysis protocol included:

\begin{itemize}
\item Random sampling of 1,000 query-passage pairs from qrels.dev.tsv
\item Consistent UMAP parameters across all visualizations
\item Separate encoding of queries and positive passages
\item Visual assessment of clustering and separation patterns
\end{itemize}

This comprehensive methodology ensures that our investigation provides reliable, reproducible insights into the phenomenon of fine-tuning failures on saturated benchmarks.
